\section{Thông tin nhóm}
\subsection{Thành viên}
\begin{longtblr}[caption={Thông tin thành viên trong nhóm}]{colspec={lc},vlines,hlines}
    \SetRow{c,font={\bfseries}} Họ và tên & Mã số sinh viên\\
    Huỳnh Thiên An & 23120001\\
    Nguyễn Ngọc Tin & 23120014\\
    Hoàng Ngọc Quí & 23120077\\
    Nguyễn Duy Bảo & 23120113
\end{longtblr}

\subsection{Phân chia công việc}
\begin{longtblr}[caption={Phân chia công việc}]{colspec={ccccc},vlines,hlines}
    \SetRow{c,font=\bfseries} \SetCell[c=2]{}\diagbox{Công việc}{Thành viên} & & H.~N.~Quí & N.~N.~Tin & N.~D.~Bảo & H.~T.~An\\
    \SetCell[c=2]{} CNF & &100\% &  &  & \\
     \SetCell[c=2]{} Sinh CNF & && 100\% & & \\	
    \SetCell[r=6]{} Thuật toán & Py-Sat & & &100\%\\
    & A* & & & 100\% \\
    & A* graph & & & 100\% \\
    & Backtracking & & & & 100\% \\
    & Backtracking graph & & & & 100\% \\
    & Brute force & 100\% \\
    \SetCell[c=2]{} Thực nghiệm và phân tích & & &100\% \\
    \SetCell[c=2]{} Báo cáo & &100\% & 100\% &  & \\
    \SetCell[c=2]{} Demo & & & & & 100\% \\
\end{longtblr}

\subsection{Mức độ hoàn thành}
Bảng~\ref{tab:completion-rate} mô tả nhóm tự đánh giá mức độ hoàn thành các yêu cầu trong đồ án.

\begin{table}[htbp]
	\centering
	\caption{Mức độ hoàn thành đồ án.}
	\label{tab:completion-rate}
\begin{tblr}{colspec={cX[l]c},vlines,hlines,rowhead=1}
    \SetRow{c,font=\bfseries} No & Tiêu chí & Mức độ hoàn thành \\
    1 & Mô tả đúng nguyên lý logic để tạo CNF & 100\% \\
    2 & Tự động sinh CNF & 100\% \\
    3 & Sử dụng thư viện PySAT để giải CNF chính xác & 100\% \\
    4 & Triển khai thuật toán A* để giải CNF không dùng thư viện & 100\% \\
    5 & Triển khai thuật toán bổ sung để so sánh với A*: \newline
        1) Thuật toán Brute-force so sánh tốc độ \newline
        2) Thuật toán Backtracking so sánh tốc độ & 100\% \\
    6 & Báo cáo và phân tích: \newline
        1) Viết báo cáo chi tiết \newline
        2) Phân tích kỹ và thực nghiệm \newline
        3) Cung cấp ít nhất 10 bộ test với kích thước khác nhau (7 $\times$ 7, 9 $\times$ 9, 11 $\times$ 11, 13 $\times$ 13, 17 $\times$ 17, 20 $\times$ 20) \newline
        4) So sánh kết quả và hiệu năng & 100\%
\end{tblr}
\end{table}


