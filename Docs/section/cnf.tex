\section{Mô hình hóa bài toán}
Bài toán Hashiwokakero được mô hình hóa thành bài toán thỏa mãn mệnh đề (SAT) và công thức dạng chuẩn hội (CNF) được sử dụng để biểu diễn luật chơi.

\subsection{Định nghĩa biến logic}
Giả sử ta có \(n \in \mathbb{N}^{\ast}\) hòn đảo, được đánh số lần lượt 
\[
1,\, 2,\, \ldots,\, n.
\]
Mỗi đảo \(i\) được gán một giá trị \(\deg(i)\), biểu diễn số cầu cần được nối đến đảo đó
(trong đó \(i = 1, 2, \ldots, n\)).

\medskip

Để mô hình hóa bài toán dưới dạng logic mệnh đề, ta giới thiệu các biến Boolean sau.
\paragraph{Biến số lượng cầu}
Ta định nghĩa biến Boolean
\[
C_{i,j}^k, \qquad i < j,\quad k \in \{1,2\},
\]
với ý nghĩa: giữa hai đảo \(i\) và \(j\) tồn tại đúng \(k\) cầu.

\paragraph{Biến cây khung}
Để mô hình hóa tính liên thông của toàn bộ hệ thống cầu, ta định nghĩa thêm hai họ biến:
\[
Par(i,j) \quad (i \neq j), \qquad Desc(i,j) \quad (i \neq j),
\]
trong đó:
\begin{itemize}
    \item \(Par(i,j)\): \(j\) là cha của \(i\) trong cây khung có gốc tại đảo \(1\),
    \item \(Desc(i,j)\): \(j\) là tổ tiên của \(i\) trong cây khung đó.
\end{itemize}

\subsection{Các ràng buộc logic}

\subsubsection{Giới hạn số cầu giữa hai đảo}

Hai đảo chỉ có thể nối bằng 0, 1 hoặc 2 cầu:
\[
\forall i<j:\qquad \lnot C_{i,j}^1 \lor \lnot C_{i,j}^2.
\]

\subsubsection{Cầu chỉ thẳng đứng hoặc thẳng ngang}

Với hai đảo \(i\) và \(j\) không cùng hàng hay không cùng cột:
\[
\lnot C_{i,j}^1 \land \lnot C_{i,j}^2.
\]

\subsubsection{Cầu không đi xuyên qua đảo khác}

Nếu tồn tại đảo \(k\) nằm giữa \(i\) và \(j\) trên cùng hàng hoặc cột:
\[
\lnot C_{i,j}^1 \land \lnot C_{i,j}^2.
\]

\subsubsection{Cầu không được cắt nhau}

Với hai cặp đảo \((i,j)\) và \((k,l)\) có đường cầu cắt nhau:
\[
(\lnot C_{i,j}^{a} \;\lor\; \lnot C_{k,l}^{b}), \qquad a,b \in \{1,2\}.
\]


\subsubsection{Tổng số cầu nối với một đảo bằng đúng số được ghi trên đảo}

\paragraph{Tập hợp các cầu tiềm năng của một đảo}
Ký hiệu tập các biến cầu liên quan đến đảo \(i\):
\[
B(i) = \{\, C_{i,j}^1,\, C_{i,j}^2,\, C_{j,i}^1,\, C_{j,i}^2 \mid j \neq i \,\}.
\]

Ràng buộc tổng bậc đúng bằng \(\deg(i)\) tương đương hai điều kiện:

\paragraph{Không vượt quá bậc yêu cầu:}
\[
\bigwedge_{\substack{S \subseteq B(i) \\ |S| = \deg(i)+1}}
\left( \bigvee_{x \in S} \lnot x \right).
\]

\paragraph{Không thiếu so với bậc yêu cầu:}
\[
\bigwedge_{\substack{S \subseteq B(i) \\ |S| = |B(i)|+1-\deg(i)}}
\left( \bigvee_{x \in S} x \right).
\]

\subsubsection{Liên thông: ràng buộc về cây khung}

\paragraph{Gốc không có cha:}
\[
\forall j \neq 1:\quad \lnot Par(1,j).
\]

\paragraph{Mỗi đỉnh khác gốc có đúng một cha:}
\[
\forall i \neq 1:\quad
\left(
\bigwedge_{i \neq j,\, j \neq k\,,k \neq i} 
(\lnot Par(i,j) \lor \lnot Par(i,k))
\right)
\;\land\;
\left( \bigvee_{j \neq i} Par(i,j) \right).
\]

\paragraph{Cha và con phải có cầu nối:}
\[
\forall i,j:\quad \lnot Par(i,j) \lor C_{i,j}^1 \lor C_{i,j}^2.
\]

\paragraph{Mối quan hệ cha–con kéo theo quan hệ tổ tiên:}
\[
\forall i \neq j:\quad \lnot Par(i,j) \lor Desc(i,j).
\]

\paragraph{Tính bắc cầu của tổ tiên:}
\[
\forall i, \forall j, \forall k,\; i \neq j,\; j \neq k,\; k \neq i:\quad 
\lnot Par(i,j) \;\lor\; \lnot Par(j,k) \;\lor\; Par(i,k).
\]

\paragraph{Không tồn tại vòng trong cây khung:}
\[
\forall i \neq j:\quad \lnot Par(i,j) \lor \lnot Par(j,i).
\]